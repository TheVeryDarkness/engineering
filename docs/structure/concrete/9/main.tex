\begin{enumerate}
    \item 截面几何参数$b=\SI{100}{\mm}$,$h=\SI{300}{\mm}$,$h_f^\prime=\SI{50}{\mm}$,$a_s = 25+16/2=\SI{33}{\mm}$,$h_0=h-a_s=\SI{267}{\mm}$,$c=\SI{25}{\mm}$。
          \par 根据表5-1,$b_f^\prime=\SI{500}{\mm}$不大于$\frac{l_0}{3}=\SI{1933}{\mm}$、$b+s_n=100+400=\SI{500}{\mm}$和$b+12h_f^\prime=100+12*50=\SI{700}{\mm}$,无需折减。
          \par 根据式11-32,受压区翼缘加强系数
          $$\gamma_f^\prime = \frac{(b_f^\prime - b) h_f^\prime}{b h_0} = \frac{500-100}{100*267} = 0.01498$$
          \par 截面力学参数$E_c = \SI{2.80e4}{\MPa}$,$E_s=\SI{2.0e5}{\MPa}$,$bh=\SI{3e4}{\mm^2}$,$A_s=2\frac{\pi}{4}16^2=\SI{402.1}{\mm^2}$,$\rho=\frac{A_s}{bh}=0.01340$,$\alpha_E=\frac{E_s}{E_c}=7.143$,$f_{tk}=\SI{1.78}{\MPa}$。
          \par 根据式11-31,开裂截面内力臂长度系数
          $$\gamma_s = 1 - 0.4\frac{\sqrt{\alpha_E\rho}}{1+2\gamma_f^\prime} = 1 - 0.4 * \frac{\sqrt{7.143*0.01340}}{1+2*0.01498} = 0.8798$$
          \par 所受外力为标准组合弯矩$M_k = \SI{18}{\kN.\m}$,准永久组合弯矩$M_q = \SI{14}{\kN.\m}$。
          \par 根据式11-30,裂缝截面处的钢筋应力
          $$\sigma_s = \frac{M_q}{A_s\gamma_sh_0} = \frac{14}{402.1*0.8798*267}=\SI{148.2}{\MPa}$$
          \par 根据式11-21,受拉有效面积
          \begin{align*}
              A_{te} = 0.5bh = 0.5*100*300 = \SI{15000}{\mm^2}
          \end{align*}
          \par 于是
          $$\rho_{te}=\frac{A_s}{A_{te}}=0.02681$$
          \par 根据式11-40,裂缝间纵向受拉钢筋应变不均匀系数
          \begin{align*}
              \psi & = 1.1 - 0.65 \frac{f_{tk}}{\rho_{te}\sigma_s} \\
                   & = 1.1-0.65*\frac{1.78}{0.02681*144.8}         \\
                   & = 0.8088
          \end{align*}
          \par 根据式11-28,纵向受拉钢筋的等效直径
          $$d_{eq}=\frac{2*16^2}{2*1.0*16}=\SI{16}{\mm}$$
          \par 根据表11-3,受弯构件的受力特征系数
          \begin{align*}
              \alpha_{cr} = 1.5 \times 1.66 \times 0.77 \times 1.0 = 1.9
          \end{align*}
          \par 根据式11-27,最大裂缝宽度
          \begin{align*}
              w_{max} & = \alpha_{cr} \psi \frac{\sigma_s}{E_s} (1.9 c + 0.08 \frac{d_{eq}}{\rho_{te}}) \\
                      & = 1.9 * 0.8088 * 148.2 / 200000 * (1.9 * 25 + 0.08 * 16 / 0.02681)              \\
                      & = \SI{0.1084}{\mm}
          \end{align*}
          \par 题中没有给出最大裂缝宽度限值或结构构件的裂缝控制等级,但对于钢筋混凝土结构,无论是一类还是二类或三类,$w_{max}=\SI{0.1084}{\mm}$都能满足\gb{}的要求。
    \item 截面几何参数$b=\SI{200}{\mm}$,$b=\SI{600}{\mm}$,$b_f^\prime=\SI{400}{\mm}$,$h_f^\prime=\SI{100}{\mm}$,$a_s=\SI{60}{\mm}$,$h_0=h-a_s=\SI{540}{\mm}$,$A_s=\SI{2945}{\mm^2}$。
          \par 开裂截面内力臂长度系数
          $$\gamma_s = 0.87$$
          \par 所受外力为标准组合弯矩$M_k=\SI{315.5}{\kN.\m}$,准永久组合弯矩$M_q=\SI{301.5}{\kN.\m}$。
          \par 根据式11-31.裂缝截面处的钢筋应力
          $$\sigma_s = \frac{M_q}{A_s\gamma_sh_0} = \frac{301.5}{2945*0.87*540}=\SI{217.9}{\MPa}$$
          \par 截面力学参数$f_{tk}=\SI{2.01}{\MPa}$,$E_s=\SI{2.0e5}{\MPa}$。
          \par 根据式11-21,受拉有效面积
          \begin{align*}
              A_{te} = 0.5bh = 0.5*200*600 = \SI{60000}{\mm^2}
          \end{align*}
          \par 于是
          $$\rho_{te}=\frac{A_s}{A_{te}}=0.04908$$
          \par 根据式11-40,裂缝间纵向受拉钢筋应变不均匀系数
          $$\psi = 1.1 - 0.65 \frac{f_{tk}}{\rho_{te}\sigma_s} = 0.9778$$
          \par 根据式11-28,纵向受拉钢筋的等效直径
          $$d_{eq}=\frac{6*25^2}{6*1.0*25}=\SI{25}{\mm}$$
          \par 根据表11-3,受弯构件的受力特征系数
          \begin{align*}
              \alpha_{cr} = 1.9
          \end{align*}
          \par 根据式11-27,最大裂缝宽度
          \begin{align*}
              w_{max} & = \alpha_{cr} \psi \frac{\sigma_s}{E_s} (1.9 c + 0.08 \frac{d_{eq}}{\rho_{te}}) \\
                      & = 1.9 * 0.9778 * 217.9 / 200000 * (1.9 * 25 + 0.08 * 25 / 0.04908)              \\
                      & = \SI{0.1786}{\mm}
          \end{align*}
          \par 满足\gb{}的要求。
    \item 考虑钢筋表面形状的系数$C_1=1.0$,荷载长期效应影响系数$C_2=1+0.5 \times 0.545 = 1.272$,构件形式相关系数$C_3=1.0$。
          \par 由荷载短期效应组合引起的开裂截面纵向受拉钢筋的应力$\sigma_{ss} = \SI{197}{\MPa}$,$E_s = \SI{2e5}{\MPa}$,$d=\frac{4 \times 16^2 + 8 \times 32^2}{4 \times 16 + 8 \times 32} = \SI{28.8}{\mm}$。
          \par 根据式11-42,纵向受拉钢筋配筋率
          $$
              \rho = \frac{A_s+A_p}{bh_0+(b_f-b)h_f} = \frac{7238}{128*1200} = 0.04712
          $$
          \par 取$\rho=0.02$。
          \par 根据式11-41,最大裂缝宽度
          \begin{align*}
              w_{max} & = C_1C_2C_3\frac{\sigma_{ss}}{E_s}\frac{30+d}{0.28+10\rho}                    \\
                      & = 1.0 * 1.272 * 1.0 * \frac{197}{200000} * \frac{30 + 28.8}{0.28 + 10 * 0.02} \\
                      & = \SI{0.1535}{\mm}
          \end{align*}
          $w_{max} = \SI{0.1535}{\mm} < w_{lim} = \SI{0.25}{\mm}$,满足要求。
    \item $E_s = \SI{2e5}{\MPa}$,$A_s = \SI{2945}{\mm^2}$,$h_0=\SI{540}{\mm}$,$\psi=0.9778$,$E_c=\SI{3e4}{\MPa}$,$E_s=\SI{2e5}{\MPa}$,$\alpha_E=200000/30000=6.667$,$\rho=\frac{2945}{200*600}=0.02454$,$\rho^\prime=\frac{628}{200*600}=0.005233$。
          \par 根据式11-32,受压区翼缘加强系数
          $$\gamma_f^\prime = \frac{(b_f^\prime - b) h_f^\prime}{b h_0} = \frac{400-200}{200*540} = 0.001852$$。
          \par 根据式11-55,短期刚度
          \begin{align*}
              B_s & = \frac{E_sA_sh_0^2}{1.15\psi+0.2+\frac{6\alpha_E\rho}{1+3.5\gamma_f^\prime}}      \\
                  & = \frac{200000*2945*540^2}{1.15*0.9778+0.2+\frac{6*6.667*0.02454}{1+3.5*0.001852}} \\
                  & = \SI{7.468e13}{\N.\mm^2}
          \end{align*}
          \par 而$\frac{\rho^\prime}{\rho}=0.2132$,在该范围内$\theta=2-0.4\frac{\rho^\prime}{\rho}=1.915$,准永久组合刚度$B=\frac{B_s}{\theta}=\frac{\num{7.468e13}}{1.915}=\SI{3.890e13}{\N.\mm^2}$。挠度
          \begin{align*}
              f & = \frac{5}{48}\frac{M_ql_0^2}{B}                          \\
                & = \frac{5}{48}\frac{\num{301.5e6}*6000^2}{\num{3.890e13}} \\
                & = \SI{29.06}{\mm}
          \end{align*}
          \par 挠度限值$\frac{l_0}{200}=\frac{6000}{200}=\SI{30}{\mm}>\SI{29.06}{\mm}$,刚好满足要求。
    \item 短期刚度
          \begin{align*}
              B_s & = \frac{E_sA_sh_0^2}{1.15\psi+0.2+\frac{6\alpha_E\rho}{1+3.5\gamma_f^\prime}}      \\
                  & = \frac{200000*402.1*267^2}{1.15*0.8088+0.2+\frac{6*7.413*0.01340}{1+3.5*0.01498}} \\
                  & = \SI{3.379e12}{\N.\mm^2}                                                          \\
          \end{align*}
          \par 准永久组合长期刚度
          \begin{align*}
              B & = \frac{B_s}{\theta}       \\
                & = \frac{\num{3.379e12}}{2} \\
                & = \SI{1.690e12}{\N.\mm^2}
          \end{align*}
          \par 挠度
          \begin{align*}
              f & = \frac{5}{48} \frac{M_ql_0^2}{B}                       \\
                & = \frac{5}{48} \frac{\num{14e6}*5800^2}{\num{1.690e12}} \\
                & = \SI{29.03}{\mm}
          \end{align*}
    \item $E_s=\SI{2e5}{\MPa}$,$A_s=\frac{\pi}{4}16^2=\SI{201.1}{\mm^2}$,$a_s=25+16/2=\SI{33}{\mm}$,$h_0=200-33=\SI{167}{\mm}$。
          \par $f_{tk}=\SI{2.01}{\MPa}$,$b=\SI{200}{\mm}$,$h=\SI{200}{\mm}$,$\rho=\frac{A_s}{bh}=\frac{201.1}{200*200}=0.005025$。
          \par $A_{te}=\frac{1}{2}bh=\SI{20000}{\mm^2}$,$\rho_{te}=\frac{A_s}{A_{te}}=\frac{201.1}{20000}=0.01006$。
          \par $E_c=\SI{3E4}{\MPa}$,$\alpha_E=6.667$。
          \par $\gamma_f^\prime=0$,$\gamma_s=1-0.4\frac{\sqrt{\alpha_E\rho}}{1+2\gamma_f^\prime}=1-0.4\frac{\sqrt{6.667*0.005025}}{1+2*0}=0.9268$。
          \par 没有给出板宽,假定弯矩为\SI{38.25}{\kN.\m/\m},则每\SI{200}{\mm}所承受弯矩为\SI{7.65}{\kN.\m}。
          \par $\sigma_s=\frac{M_q}{A_s\gamma_sh_0}=\frac{\num{7.65e6}}{201.1*0.9268*167}=\SI{245.8}{\MPa}$。
          \par $\psi=1.1-0.65\frac{f_{tk}}{\rho_{te}\sigma_s}=1.1-0.65\frac{2.01}{0.01006*245.8}=0.5716$。
          \par 短期刚度
          \begin{align*}
              B_s & = \frac{E_sA_sh_0^2}{1.15\psi+0.2+\frac{6\alpha_E\rho}{1+3.5\gamma_f^\prime}} \\
                  & = \frac{200000*201.1*167^2}{1.15*0.5716+0.2+\frac{6*6.667*0.005025}{1+3.5*0}} \\
                  & = \SI{1.0598e12}{\N.\mm^2}
          \end{align*}
          \par $\theta=2$,长期刚度$B = \frac{B_s}{\theta} = \SI{5.299e11}{\N.\mm^2}$。
          \par 弯矩$M=\frac{ql^2}{2}$,挠度$\delta=\frac{\frac{2}{3}l*\frac{1}{3}l*\frac{1}{2}ql^2}{EI}=\frac{ql^4}{9EI}=\frac{2Ml^2}{9EI}$,挠度
          \begin{align*}
              f & = \frac{2}{9}\frac{M_kl_0^2}{B}                          \\
                & = \frac{2}{9} \frac{\num{7.65e6}*3000^2}{\num{5.299e11}} \\
                & = \SI{28.87}{\mm}
          \end{align*}
          \par 悬挑端挠度限值$\frac{l}{300}=\frac{3000}{300}=\SI{10}{\mm}<f$,不满足。
    \item $\sigma_{\rm{pc\RN{2}}}=\SI{11.71}{\MPa}$,$\sigma_{ck}=\frac{N_k}{A_0}$,要求为
          \begin{align*}
              \sigma_{\rm{{ck}}} - \sigma_{\rm{pc\RN{2}}} & \le f_{\rm{tk}}                          \\
              \frac{N_k}{A_0}                             & \le f_{\rm{tk}} + \sigma_{\rm{pc\RN{2}}} \\
                                                          & = 2.39 + 11.71                           \\
                                                          & = \SI{14.20}{\MPa}
          \end{align*}
          \par $E_{\rm{c}}=\SI{3.25e4}{\MPa}$,$E_{\rm{p}}=\SI{2.05e5}{\MPa}$,$\alpha_{\rm{E_p}}=6.305$
          \par $A_{\rm{p}}=9\frac{\pi}{4}9^2=\SI{572.6}{\mm^2}$,$bh=\SI{40000}{\mm^2}$,$A_{\rm{0}}=bh+(\alpha_{\rm{E_p}}-1)A_{\rm{p}}=\SI{43038}{\mm^2}$。
          \par 所以$N_{\rm{k}} \le \SI{611.1}{\kN}$。
    \item \par $A_{\rm{0}}=\SI{53946}{\mm^2}$,$\sigma_{\rm{pc\RN{2}}}=\SI{8.417}{\MPa}$。
          \par $\sigma_{\rm{ck}}=\frac{N_{\rm{k}}}{A_{\rm{0}}}=\frac{\num{600e3}}{53946}=\SI{11.12}{\MPa}$。
          \par $\sigma_{\rm{ck}}-\sigma_{\rm{pc\RN{2}}}=\SI{2.703}{\MPa}>f_{\rm{tk}}=\SI{2.39}{\MPa}$。不满足。
    \item $\sigma_{\rm{l\RN{1}}}=\SI{10.5}{\MPa}$,$\sigma_{\rm{con}}=\SI{889}{\MPa}$,$\sigma_{\rm{l\RN{2}}}=19+53=\SI{72}{\MPa}$。
          \par $E_{\rm{p}}=\SI{2.05e5}{\MPa}$,$E_{\rm{c}}=\SI{3.25e4}{\MPa}$,$\alpha_E=6.308$,$A_{\rm{p}}=3\frac{\pi}{4}9^2=\SI{190.8}{\mm^2}$,$A_{\rm{0}}=bh+(\alpha_{\rm{E}}-1)A_{\rm{p}}=\SI{46012}{\mm^2}$,$W_{\rm{0}}=\frac{1}{3}bh^2+(\alpha_{\rm{E}}-1)A_{\rm{p}}(h/2-a_{\rm{s}})=\SI{4.616e6}{\mm^3}$。
          \par $\sigma_{\rm{pc\RN{2}}}=(889-10.5-72)*190.8/46012=\SI{3.344}{\MPa}$。
          \par $\sigma_{\rm{ck}}=\frac{M_{\rm{q}}}{W_{\rm{0}}}=\frac{\num{8e6}}{\num{4.616e6}}=\SI{1.733}{\MPa}$
          \par $\sigma_{\rm{ck}}-\sigma_{\rm{pc\RN{2}}}=\SI{-1.611}{\MPa}$,满足。
\end{enumerate}