\documentclass{article}
\usepackage{amsmath}
\usepackage{enumerate}
\usepackage{geometry}
\usepackage{hyperref}
\usepackage{textcomp}
\usepackage[UTF8]{ctex}

\geometry{a4paper, scale=0.7}

\begin{document}
\title{Basic Principle of Concrete Structure}
\author{H.}
\maketitle
\newpage
\tableofcontents
\newpage
\section{绪论}
\subsection{混凝土结构的一般概念和特点}
\subsubsection{钢筋混凝土结构的一般概念}
\par混凝土的抗拉强度一般仅为抗压强度的$1/10$左右。钢筋和混凝土结合在一起共同工作,则能各尽其用。钢筋提高了梁的承载能力、变形能力,还使梁在破坏前能给人明显的预告。
\par施工时,一般先按构件形状尺寸制作模板,再将钢筋放入模板中适当的位置固定,最后浇筑混凝土,待混凝土结硬成型并达到一定强度时除去模板。
\subsubsection{钢筋和混凝土共同工作的原因}
\begin{itemize}
    \item 粘结性能良好;
    \item 温度线膨胀系数很接近;
    \item 混凝土包裹钢筋,可防止钢筋过早腐蚀或高温软化。
\end{itemize}
\subsubsection{预应力混凝土结构的一般概念}
\par预压应力抵消外部荷载产生的拉应力,提高抗裂性能。
\subsubsection{混凝土结构的组成}
\par板、梁、柱、墙和基础等。
\subsubsection{混凝土结构的优缺点}
\begin{itemize}
    \item 优点:
          \begin{itemize}
              \item 良好的耐久性。
              \item 良好的耐火性。
              \item 良好的整体性。
              \item 良好的可模性。
              \item 可就地取材。
              \item 节约钢材。
          \end{itemize}
    \item 缺点:
          \begin{itemize}
              \item 自重大;
              \item 不利于大跨、高层抗震;
              \item 易开裂;
              \item 现浇耗费大量模板;
              \item 施工受季节性影响较大;
              \item 隔热隔声性能较差等。
          \end{itemize}
\end{itemize}
\subsection{混凝土结构的发展}
\subsubsection{混凝土结构的诞生}
\subsubsection{混凝土结构材料方面的发展}
\subsubsection{混凝土结构材料体系的发展}
\subsubsection{混凝土结构的模型试验技术和计算机仿真技术}
\subsection{混凝土结构的应用}
\subsection{本课程的特点和学习方法}
\section{钢筋与混凝土材料的基本性能}
\subsection{钢筋的强度和变形}
\subsubsection{钢筋的形式和成型}
\par混凝土结构构件中配置的钢筋可以是劲性钢筋或柔性钢筋。前者是型钢焊成的骨架。混凝土结构中更多采用柔性钢筋,柔性钢筋以下简称钢筋。
\par钢筋按其表面形状可分为光圆钢筋和带肋钢筋(或称变形钢筋)。
\par带肋钢筋是在钢筋表面轧制纵向肋纹(可不带)和横向斜肋纹。肋纹有螺纹、人字纹、月牙纹等多种形式。带肋钢筋截面随纵轴长度变化,直径为以重量计的当量直径。钢筋表面的肋有利于钢筋与混凝土两种材料的结合。
\begin{table}[htbp]
    \caption{钢筋的直径(单位为$mm$)}
    \begin{center}
        \begin{tabular}{|c|ccccccccc|cccccc|}
            \hline
            光圆钢筋 & 6 & 8 & 10 & 12 & 14 & 16 & 18 & 20 & 22 &    &    &    &    &    &    \\
            \hline
            带肋钢筋 & 6 & 8 & 10 & 12 & 14 & 16 & 18 & 20 & 22 & 25 & 28 & 32 & 36 & 40 & 50 \\
            \hline
        \end{tabular}
    \end{center}
\end{table}
\par直径较小的钢筋也称钢丝,通常为光圆外形,若在表面机械刻痕以提高粘结力,则称刻痕钢丝。多股钢丝捻在一起形成的钢绞线也可以作为配筋。
\par浇筑前可以将布置在构件中的各种钢筋用绑扎或焊接的方法做成钢筋骨架或钢筋网片,就可以保持其相对位置,有利于发挥钢筋的强度。
\begin{itemize}
    \item 为了防止受拉的光面钢筋滑动,应在其端部设弯钩。
    \item 为了设计的要求\footnote{比如受拉、受压侧的互换等。},有时还需在钢筋中间区段弯转。
    \item 受压的光面钢筋端部可不设弯钩,因为钢筋受压时周围的混凝土会约束其截面膨胀或偏移,进而有利于阻止其滑动。
    \item 带肋钢筋表面有齿肋花纹,与混凝土有很好的结合性能,端部可不设弯钩。有时为了满足锚固长度的要求而必须设置时,采用较易成形的直角型弯钩。
    \item 焊接方法制成的钢筋骨架或钢筋网片,能与混凝土较好地结合,钢筋端部可不设置弯钩。
\end{itemize}
\par为了保证钢筋在加工、使用时不开裂、弯断和脆断,通常用冷弯试验\footnote{用外力使钢筋围绕一定直径的辊轴弯转,在达到规定的弯转角度后钢筋不能出现裂纹或断裂。}来检验其韧性和内部质量。焊接的钢筋骨架和网片省工省料、适用于工业化批量生产、装配式结构的生产,能减少工作量,加快施工进度。需要焊接的钢筋应具有较好的可焊性。
\subsubsection{单调荷载下钢筋的强度和变形}
\par常规的荷载试验通常采用单调加载。
\par热轧低碳钢和普通热轧低合金钢等应力--应变关系随应变增加表现为:线性--比例极限--轻微非线性--流幅(屈服台阶)--强化--极限点--颈缩--断裂。
\par高碳钢应力--应变关系中没有明显的屈服点和流幅,一般取残余应变为$0.2\%$时对应的应力$\sigma_{0.2}$作为钢筋的条件屈服强度。但冶金系统产品标准中规定屈服强度$\sigma_{0.2}$不得小于极限抗拉强度$\sigma_b$的$85\%$,因此实际应用中可取极限抗拉强度$\sigma_b$的$85\%$作为条件屈服点。
\par具有明显流幅的钢材统称为软钢,无明显流幅的钢材统称为硬钢。流幅$\varepsilon_{sh}$和极限应变$\varepsilon_{su}$是较重要的塑性指标。软钢的应力--应变关系通常采用理想弹塑性模型(两折线模型),即
$$
    \sigma_s =
    \left\{ \begin{aligned}
        E_s \varepsilon_s & (\varepsilon_s \leq \varepsilon_y) \\
        f_y               & (\varepsilon_s >\varepsilon_y)
    \end{aligned} \right.
$$
\par钢筋可根据化学成分分为碳素钢、普通合金钢两大类,根据加工方式分为热轧钢筋、冷拉钢筋和热处理钢筋三大类。
\par钢筋强度标准值取强度平均值减去两倍标准差时,保证率可达到97.73\%。《混凝土结构设计规范》(GB 50010)采用概率极限状态设计法,于是钢筋强度设计值等于钢筋强度标准值除以钢筋材料分项系数$\gamma_s=1.1$;还规定,对结构构件进行承载力设计时,采用钢筋强度设计值,进行变形和裂缝宽度验算时,采用钢筋强度标准值。
\subsubsection{钢筋的冷加工和热处理}
\par冷拉或冷拔的冷加工方法可以提高热轧钢筋的强度。
\par冷拉加工是在常温下把软钢拉伸到超过屈服强度的某一应力值,然后卸去拉力。此时将产生残余应变,不再有明显的屈服台阶,塑性变差,抗拉强度提高,抗压强度不变。
\par如果卸去拉力后在自然条件下放置一段时间后再进行拉伸,则屈服点可进一步提高,这种现象称为时效硬化。原材料强度越高,冷拉加工提高幅度越小。温度达700\textcelsius时,冷拉的效果消失,因此需要焊接和冷拉的钢筋应先焊接再冷拉。
\par冷拔加工是用强力将钢筋拔过硬质合金拔丝模上比其直径稍小的锥形孔,这时钢筋会在轴向拉力和横向挤压力的同时作用下产生塑性变形,横截面减小,长度增加,内部结构发生变化,抗拉抗压强度同时提高。多次冷拔后,钢筋的塑性明显降低,没有明显的屈服点和流幅。
\subsubsection{钢筋的徐变和松弛}
\par钢筋在较高应力的持续作用下,应变会随时间增长继续增加,这种现象称为徐变;若保持受力钢筋长度不变,则应力会随时间增长降低,这种现象称为松弛。
\par通常,初始应力大,徐变或松弛损失也大;冷拉热轧钢筋的徐变或松弛较冷拔低碳钢丝、碳素钢丝和钢绞线更低;温度增加,徐变或松弛也增大。
\subsubsection{重复和反复荷载下钢筋的强度和变形}
\par重复荷载指同一方向反复加载卸载。应力--应变关系曲线的包络图与单调荷载下的应力--应变关系曲线几乎相同。
\par反复荷载指在相反方向交替加载卸载。若加载超过屈服应变后卸载,应力--应变关系曲线沿加载直线平行方向下行,再反向加载时,立即塑性变形,弹性极限相对单调荷载下降低,这种现象称为“包兴格效应”。
\par钢筋承受周期性的重复荷载后,可能在未达到单调加载时的强度时破坏,这种现象称为疲劳破坏。疲劳强度指在规定应力范围内,经受一定次数循环荷载后发生疲劳破坏的最大应力值;它与应力幅限值(一次循环应力中最大与最小应力间的差值)、钢筋表面的形状、钢筋的直径、强度、加工使用的环境和加载的频率等有关。疲劳应力比值是指同一层钢筋的最小应力与最大应力的比值。
\subsection{混凝土的强度和变形}
\par普通混凝土是由水泥、砂、石材料用水拌合硬化后形成的人工石材,一种复杂的多相复合材料。其中的砂、石、水泥胶块中的晶体、未水化的水泥颗粒组成了弹性骨架,水泥胶块中的凝胶、孔隙和结合界面的初始微裂缝则为塑性变形提供可能。孔隙、初始裂缝等先天缺陷往往是混凝土受力破坏的起因,而且微裂缝的开展对力学性能有极为重要的影响。
\subsection{混凝土立方体受压}
\par混凝土立方体抗压强度是确定混凝土强度等级的依据,《混凝土结构设计规范》(GB 50010)规定混凝土强度按立方体抗压强度标准值$f_{cu,k}$确定。
\par混凝土强度与水泥强度等级、水灰比关系很大,也受骨料的性质、混凝土的级配、混凝土成形方法、硬化时的环境条件及混凝土的龄期等影响;试验时试件的大小形状、试验方法和加载速率也会影响测得的强度。
\subsubsection{混凝土轴心受压}
\par混凝土棱柱体试件试验得到的轴心抗拉强度值能更好地反映混凝土结构构件中混凝土实际抗压能力。混凝土轴心抗压强度设计值$f_c$是轴心抗压强度标准值$f_{ck}$除以混凝土材料分项系数$\gamma_c=1.4$得到的。而《混凝土结构设计规范》(GB 50010)偏于安全地表示
$$
f_{ck} = 0.88 \alpha_1 \alpha_2 f_{cu,k}
$$
\par其中,$\alpha_1$为棱柱体强度与立方体强度之比,$\alpha_2$为高强度混凝土的脆性折减系数。
\appendix
\newcommand{\ans}{\par答:}
\section{思考题}
\begin{enumerate}
    \item
          \begin{enumerate}[1.]
              \item 钢筋和混凝土共同工作的基础是什么?
                    \ans有三点,两者之间良好的粘结性能、相近的温度线膨胀系数、混凝土对钢筋的包裹带来的对钢筋锈蚀或高温软化的抑制作用。
              \item 与素混凝土相比,钢筋混凝土梁有哪些优势?
                    \ans提高梁的承载能力、变形能力,并使其在破坏前能预警。
              \item 与钢筋混凝土梁相比,预应力混凝土梁有哪些优势?
                    \ans预应力产生的预压应力会抵消(全部或大部分)外部荷载产生的拉应力,提高抗裂性能。
              \item 与其他结构相比,混凝土结构有哪些特点?
                    \ans优点有:良好的耐久性、耐火性、整体性、可模性,可就地取材,节约钢材。缺点有:自重大,对大跨、高层抗震不利,易开裂,现浇需耗费大量模板,施工受季节性影响较大,隔热隔声性能较差等。
          \end{enumerate}
\end{enumerate}
\section{习题}
\end{document}