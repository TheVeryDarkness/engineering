\documentclass{article}
\usepackage[UTF8]{ctex}

\begin{document}
\title{Basic Principle of Concrete Structure}
\author{H.}
\maketitle
\tableofcontents
\section{绪论}
\subsection{混凝土结构的一般概念和特点}
\subsubsection{钢筋混凝土结构的一般概念}
\par混凝土的抗拉强度一般仅为抗压强度的$1/10$左右。钢筋和混凝土结合在一起共同工作,则能各尽其用。钢筋提高了梁的承载能力、变形能力,还使梁在破坏前能给人明显的预告。
\par施工时,一般先按构件形状尺寸制作模板,再将钢筋放入模板中适当的位置固定,最后浇筑混凝土,待混凝土结硬成型并达到一定强度时除去模板。
\subsubsection{钢筋和混凝土共同工作的原因}
\begin{itemize}
    \item 粘结性能良好;
    \item 温度线膨胀系数很接近;
    \item 混凝土包裹钢筋,可防止钢筋过早腐蚀或高温软化。
\end{itemize}
\subsubsection{预应力混凝土结构的一般概念}
\par预压应力抵消外部荷载产生的拉应力,提高抗裂性能。
\subsubsection{混凝土结构的组成}
\par板、梁、柱、墙和基础等。
\subsubsection{混凝土结构的优缺点}
\begin{itemize}
    \item 优点:
          \begin{itemize}
              \item 良好的耐久性。
              \item 良好的耐火性。
              \item 良好的整体性。
              \item 良好的可模性。
              \item 可就地取材。
              \item 节约钢材。
          \end{itemize}
    \item 缺点:
          \begin{itemize}
              \item 自重大;
              \item 不利于大跨、高层抗震;
              \item 易开裂;
              \item 现浇耗费大量模板;
              \item 施工受季节性影响较大;
              \item 隔热隔声性能较差等。
          \end{itemize}
\end{itemize}
\subsection{混凝土结构的发展}
\subsubsection{混凝土结构的诞生}

\end{document}